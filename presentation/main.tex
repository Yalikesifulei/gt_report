\documentclass[10pt,pdf]{beamer}

\usefonttheme[onlymath]{serif}
\usepackage{amsmath,amsfonts,amssymb,amsthm,mathtools}
\usepackage[T2A]{fontenc}
\usepackage[utf8]{inputenc}
\usepackage[english,ukrainian]{babel}
\usepackage{ragged2e}
\justifying
\addtobeamertemplate{navigation symbols}{}{%
    \usebeamerfont{footline}%
    \usebeamercolor[fg]{footline}%
    \hspace{1em}%
    \insertframenumber/\inserttotalframenumber
}

\usepackage{tikz}
\usepackage{pgfplots}
\usepgfplotslibrary{fillbetween}
\pgfplotsset{compat=1.17}

\usetikzlibrary{patterns}
\usetikzlibrary{shapes.geometric}
\usetikzlibrary{arrows.meta}
\usetikzlibrary{shapes.geometric}
\usetikzlibrary{positioning,arrows}
\usetikzlibrary{decorations.markings}
\usetikzlibrary{calc}
\tikzset{
	block/.style={
		draw, 
		rectangle, 
		minimum height=1cm, 
		minimum width=3cm, align=center
	}, 
	line/.style={->,>=stealth'}}
\tikzset{>=latex}

\newcommand{\R}{\mathbb{R}}
\newcommand{\vf}{\varphi}
\renewcommand{\d}[1]{\dot{#1}}
\newcommand{\dd}[1]{\ddot{#1}}
\newcommand{\E}{\mathcal{E}}
\newcommand{\T}{\mathcal{T}}
\newcommand{\bvec}[1]{\boldsymbol{\rm #1}}
\renewcommand{\l}{\left}
\renewcommand{\r}{\right}
\newcommand{\norm}[1]{\left\Vert #1 \right\Vert}
\newcommand{\intl}{\int\limits}
\newcommand{\suml}{\sum\limits}
\newcommand{\dotprod}[2]{\l< #1, #2\r>}
\newcommand{\Lap}[1]{\mathcal{L}\l\{#1\r\}}
\newcommand{\LapInv}[1]{\mathcal{L}^{-1}\l\{#1\r\}}
\def\setdif{\mathbin{\ooalign{\hss\raise1ex\hbox{.}\hss\cr
  \mathsurround=0pt$-$}}}

\usepackage{graphicx}
\graphicspath{ {./pics/} }

\usetheme{Luebeck}
\title{Основи диференціальних ігор}
\author{Н. Фордуй, О. Галганов}


\date{}
\begin{document}
    \begin{frame}
        \titlepage
    \end{frame}
    \begin{frame}
        \frametitle{Основні поняття}
    
        Рiшення, що їх приймають гравцi, полягають у виборi так званих \textbf{керувань}, вiд яких залежать \textbf{фазовi координати}: 
        їх значення у будь-який момент часу повнiстю визначає хiд гри, характеризуючи положення гравцiв у деякому просторi — \textbf{фазовому просторi}. 

        Поточнi значення фазових координат завжди вiдомi
        гравцям — тобто, це iгри з повною iнформацiєю. Невiдомим зазвичай є
        характер їх змiни: тобто, \textbf{керування} фазовими змiнними гравцями.
        \begin{block}{Приклад}
            Положення матеріальної точки на площині описується двома координатами $x_1$ та $x_2$. 
            Нехай швидкість руху точки є сталою $v$, а гравець обирає напрямок швидкості $\vf$ та може змінювати його у будь-який момент часу --- тобто, $\vf$
            є керуванням. Тоді рух точки описується системою диференціальних рівнянь
            \begin{gather*}
                \begin{cases}
                    \d{x_1} = v \cos \vf \\
                    \d{x_2} = v \sin \vf
                \end{cases}
            \end{gather*}
        \end{block}
    \end{frame}
    \begin{frame}
        \frametitle{Приклад}
    
        Геометричне положення автомобіля на декартовій площині описується трьома фазовими координатами:
        $x_1, x_2$ --- положення деякої точки автомобіля, $x_3$ --- кут, який утворює вісь вздовж автомобіля
        з деяким фіксованим напрямком --- наприклад, $x_1$.
        \begin{center}
            \begin{tikzpicture}
                \draw [->, thick] (-0.5, 0) -- (5, 0);
                \draw [->, thick] (0, -0.5) -- (0, 3);
                \node [below] at (5, 0) {$x_1$};
                \node [left] at (0, 3) {$x_2$};
                \filldraw [fill=lightgray, rounded corners, rotate around={-40:(1,1)}] (1, 1) rectangle (1.7, 2.4);
                \fill [rotate around={-40:(1,1)}] (1.35, 2.4) circle [radius=2pt];
                \draw [->, dashed, rotate around={-40:(1,1)}] (1.35, 0) -- (1.35, 3);
                \draw (0.83, 0) arc (0:44:0.2);
                \draw (0.93, 0) arc (0:45:0.3);
                \node [right] at (0.9, 0.22) {$x_3$};
            \end{tikzpicture}
        \end{center}
    \end{frame}
    \begin{frame}
        \frametitle{Продовження приклада}

        Нехай $A$ --- максимальне можливе прискорення автомобіля, тоді прискорення може набувати значень
        $A \vf_1$, де $\vf_1 \in [0; 1]$ і знаходиться під контролем гравця-водія. Можна ввести ще одну фазову координату $x_4$ --- швидкість автомобіля.
        Таким чином, можна ввести кривину як ще одну фазову координату $x_5$
        (фізично --- це кут повороту передніх коліс), керуванням якої є $W \vf_2$, де $\vf_2 \in [-1; 1]$, а $W$ --- максимальна швидкість зміни $x_5$.

        Система задає рух автомобіля у деякій диференціальній грі:
        \begin{gather*}
            \begin{cases}
                \d{x_1} = x_4 \cos{x_3} \\
                \d{x_2} = x_4 \sin{x_3} \\
                \d{x_3} = x_4 x_5 \\
                \d{x_4} = A \vf_1, \; \vf_1 \in [0; 1] \\
                \d{x_5} = W \vf_2, \; \vf_2 \in [-1; 1]
            \end{cases}
        \end{gather*}
    \end{frame}
    \begin{frame}
        \frametitle{Опис руху}
    
        Вважаємо, що гра відбувається у \emph{фазовому просторі} $\E$ --- деякій області в $\R^n$ та на її межі.
        Рух точки $x = \l(x_1, x_2, ..., x_n \r)$ у фазовому просторі описується системою диференціальних рівнянь
        \begin{gather*}\label{eq_1}
            \begin{cases}
                \d{x_1}(t) = f_1(x_1(t), ..., x_n(t), u_1(x, t), ..., u_P(x, t), v_1(x, t), ..., w_E(x, t)) \\
                \d{x_2}(t) = f_2(x_1(t), ..., x_n(t), u_1(x, t), ..., u_P(x, t), v_1(x, t), ..., w_E(x, t)) \\
                \dots \\
                \d{x_n}(t) = f_n(x_1(t), ..., x_n(t), u_1(x, t), ..., u_P(x, t), v_1(x, t), ..., w_E(x, t)) \\
                x_1(0) = x_1^0, x_2(0) = x_2^0, ..., x_n(0) = x_n^0
            \end{cases}
        \end{gather*}
        або, коротше,
        \begin{gather*}\label{eq_2}
            \begin{cases}
                \d{x}(t) = {f}(x(t), u(x, t), v(x, t)) \\
                x(0) = x_0
            \end{cases}
        \end{gather*}
        Ці рівняння називаються \emph{рівняннями руху}. Функції $f_j$ є заданими та вважаються достатньо гладкими.
    \end{frame}
    \begin{frame}
        \frametitle{Приклад}
    
        Якщо позначити через $(x_P, y_P)$ координати гравця $P$, через $(x_E, y_E)$ --- гравця $E$, через $w_P$ та $w_E$ їх сталі швидкості руху, 
        а керування напрямком швидкості через $u(t)$ та $v(t)$ відповідно, то отримаємо такі рівняння руху:
        \begin{gather*}
            \begin{cases}
                \d{x_P}(t) = w_P \cos u(t) \\
                \d{y_P}(t) = w_P \sin u(t) \\
                \d{x_E}(t) = w_E \cos v(t) \\
                \d{y_E}(t) = w_E \sin v(t) \\
                (x_P(0), y_P(0)) = (x_P^0, y_P^0) \\
                (x_E(0), y_E(0)) = (x_E^0, y_E^0)
            \end{cases}
        \end{gather*}
        Такий рух називається <<переслідуванням на площині з простим рухом гравців>>.
    \end{frame}
    \begin{frame}
        \frametitle{Виграші}
    
        Мета диференціальної гри визначається виграшем, який залежить від траєкторій гравців. 
        Позначимо ці траєкторії як функції від часу як $x(t)$ та $y(t)$. 
        Зауважимо, що диференціальні ігри є \emph{антагоністичними} (або ж, \emph{іграми з нульовою сумою}).

        Якщо гра триває деякий заздалегідь визначений час $T$, то виграш гравця $E$ визначається
        як $H(x(t), y(T))$, де $H : \R^n \times \R^n \to \R$ --- деяка функція (нагадаємо, що розмірність $\E$ --- $n$).

        \begin{block}{Приклади виграшів}
            \begin{enumerate}
                \item $H(x(T), y(T)) = \norm{x(T) - y(T)}$
                \item $H(x(T), y(T)) = \underset{0 \leq t \leq T}{\min} \norm{x(t) - y(t)}$
                \item $H(x(T), y(T)) = t_* = \min \l\{ t \geq 0 : (x(t), y(t)) \in \T\r\}$
            \end{enumerate}
        \end{block}
    \end{frame}
    \begin{frame}
        \frametitle{Приклад}
    
        Розглянемо переслідування на площині з простим рухом, що описується системою
        \begin{gather*}
            \begin{cases}
                \d{x_1} = u_1, \d{x_2} = u_2, \; u_1^2 + u_2^2 \leq \alpha^2 \\
                \d{y_1} = v_1, \d{y_2} = v_2, \; v_1^2 + v_2^2 \leq \beta^2 \\
                x_1(0) = x_1^0, x_2(0) = x_2^0, y_1(0) = y_1^0, y_2(0) = y_2^0
            \end{cases}
        \end{gather*}

        Якщо $\alpha > \beta$, то гравець $P$ може гарантувати
        \begin{gather*}
            \forall l \geq 0 : \min \l\{ t \geq 0 : \norm{P(t) - E(t)} \leq l\r\} < +\infty
        \end{gather*}

        Якщо $\alpha \leq \beta$, то в разі $\norm{P(0) - E(0)} > l$ для всіх $l \geq 0$ гравець $E$, рухаючись від $P$ по прямій з максимальною швидкістю,
        зможе уникнути захоплення гравцем $P$.
    \end{frame}
    \begin{frame}
        \frametitle{Поняття стратегії}
    
        \begin{block}{Означення}
            \emph{Стратегіями} у диференціальній грі є вибір керувань $u$ та $v$ як функцій від часу $t$ та
            фазових координат $x$ у системі рівнянь руху
            $$
            \begin{cases}
                \d{x}(t) = {f}(x(t), u(x, t), v(x, t)) \\
                x(0) = x_0
            \end{cases}
            $$
        \end{block}
        Керування вважаються кусково-гладкими як компроміс між забезпеченням існування розв'язку, 
        (його може не існувати у класі неперервних функцій) та його єдиності (вона може порушуватися, якщо не вимагати неперервності розв'язку).

        Позначатимемо через $\rm P$ та $\rm E$ множини кусково-неперервних стратегій (керувань) гравців $P$ та $E$. 
    \end{frame}
    \begin{frame}
        \frametitle{Ситуація}
    
        Надалі для спрощення розглядатимемо не один вектор $x$, а два вектори $x$ та $y$, що відповідатимуть руху кожного з гравців. Тоді
        систему можна записати як
        \begin{gather*}
            \begin{cases}
                \d{x}(t) = f(x(t), u(x, y, t)) \\
                \d{y}(t) = g(x(t), v(x, y, t)) \\
                x(0) = x_0, y(0) = y_0
            \end{cases}
        \end{gather*}

        \begin{block}{Означення}
            Набір $S = \l\{x_0, y_0, u(\cdot), v(\cdot) \r\}$, де $x_0, y_0$ --- початкові умови, а $u \in \rm P$, $v \in \rm E$ --- керування, 
            називається \emph{ситуацією} в диференціальній грі.
        \end{block}

        
    \end{frame}
    \begin{frame}
        \frametitle{Умова існування та єдиності траекторій}
    
        Якщо розглядати траєкторії, що залежать лише від часу $t$ та накладати на $f$ та $g$ умови
        обмеженості та ліпшицевості по $x$ та $y$, тобто
        \begin{gather*}
            \norm{f(x_1, u) - f(x_2, u)} \leq \alpha \cdot \norm{x_1 - x_2}, \\
            \norm{g(y_1, v) - g(y_2, v)} \leq \beta \cdot \norm{y_1 - y_2},
        \end{gather*}
        то за теоремою про існування та єдиність роз'язку задачі Коші, для кожної ситуації $S$ буде існувати єдина пара траєкторій $x(t), y(t)$,
        для якої 
        \begin{gather*}
            \begin{cases}
                \d{x}(t) = f(x(t), u(t)) \\
                \d{y}(t) = g(y(t), v(t)) \\
                x(0) = x_0, y(0) = y_0
            \end{cases}
        \end{gather*}
    \end{frame}
    \begin{frame}
        \frametitle{Формальне визначення виграшів та їх види}

        \begin{block}{Означення}
            Користуючись означенням ситуації, можна ввести \emph{виграш} в ситуації $S = \l\{x_0, y_0, u(\cdot), v(\cdot) \r\}$
            як функцію $K(x_0, y_0, u(\cdot), v(\cdot))$.
        \end{block}

        Наведемо строгі означення 4 видів виграшів.

        \begin{block}{Означення}
            \begin{enumerate}
                \item \emph{Термінальний виграш.} Задано деяке число $t>0$ та неперервна по $x$ та $y$ функція $H(x, y)$. Виграш в ситуації $\l\{x_0, y_0, u(\cdot), v(\cdot) \r\}$
                визначається як:
                \begin{gather*}
                    K(x_0, y_0, u(\cdot), v(\cdot)) = H(x(T), y(T))
                \end{gather*}
            \end{enumerate}
        \end{block}
    \end{frame}
    \begin{frame}
        \frametitle{Види виграшів}
        
        \begin{block}{Означення}
            \begin{enumerate}
                \setcounter{enumi}{1}
                \item \emph{Мінімальний результат}. Задано деяке число $t>0$ та неперервна по $x$ та $y$ функція $H(x, y)$. Виграш в ситуації $\l\{x_0, y_0, u(\cdot), v(\cdot) \r\}$
                визначається як
                \begin{gather*}
                    K(x_0, y_0, u(\cdot), v(\cdot)) = \underset{0 \leq t \leq T}{\min} H(x(t), y(t))
                \end{gather*}

                \item  \emph{Інтегральний виграш}. Нехай $\T$ --- деяка підмножина $\R^n \times \R^n$, $H(x, y)$ --- неперервна функція. Нехай в ситуації $\l\{x_0, y_0, u(\cdot), v(\cdot) \r\}$
                $t_*$ --- перший момент потрапляння траєкторії $(x(t), y(t))$ на $\T$.
                Тоді
                \begin{gather*}
                    K(x_0, y_0, u(\cdot), v(\cdot)) = \intl_0^{t_*} H(x(t), y(t)) dt
                \end{gather*}
                де при $t_* = +\infty$ покладається $K = +\infty$.

            \end{enumerate}
        \end{block}
    
    \end{frame}
    \begin{frame}
        \frametitle{Види виграшів}
    
        \begin{block}{Означення}
            \begin{enumerate}
                \setcounter{enumi}{3}
                \item \emph{Якісний виграш}. Нехай $\T$ та $\mathcal{L}$ --- деякі підмножини $\R^n \times \R^n$, а $t_*$ --- перший момент потрапляння траєкторії $(x(t), y(t))$ на $\T$
                в ситуації $\l\{x_0, y_0, u(\cdot), v(\cdot) \r\}$. Тоді
                \begin{gather*}
                    K(x_0, y_0, u(\cdot), v(\cdot)) = \begin{cases}
                        1, & \text{ якщо } (x(t_*), y(t_*)) \in \mathcal{L} \\
                        0, & \text{ якщо } t_* = +\infty \\
                        -1, & \text{ якщо } (x(t_*), y(t_*)) \notin \mathcal{L} \\
                    \end{cases}
                \end{gather*}
            \end{enumerate}
        \end{block}
    \end{frame}
    \begin{frame}
        \frametitle{Нормальна форма диференціальної гри}
    
        Нарешті, можна дати означення нормальної форми диференціальної гри.

        \begin{block}{Означення}
            Нормальною формою диференціальної гри $\Gamma (x_0, y_0)$, заданої на просторі стратегій $\mathrm{P} \times \mathrm{E}$, називається система
            \begin{gather*}
                \Gamma (x_0, y_0) = \l<x_0, y_0, \mathrm {P}, \mathrm{E}, K(x_0, y_0, u(\cdot), v(\cdot)) \r>
            \end{gather*}
            де $K(x_0, y_0, u(\cdot), v(\cdot))$ --- функція виграшу, визначена будь-який з чотирьох способів вище.
        \end{block}

        Кожній парі $(x_0, y_0) \in \R^n \times \R^n$ відповідає своя гра в нормальній формі, тобто, фактично,
        визначається двопараметрична сім'я ігор, що залежать від $(x_0, y_0)$.
    \end{frame}
    \begin{frame}
        \frametitle{Простий рух на площині}
    
        Розглянемо найпростіші моделі задач переслідування --- диференціальні ігри на площині з двома учасниками:
        переслідувачем $P$ та утікачем $E$, траєкторії яких відповідно позначатимемо $x(t)$ та $y(t)$.
        Під \emph{простим рухом} мається на увазі, що закони їх руху описуються системою
        \begin{gather*}
            \begin{cases}
                \d{x} = u, & \norm{u} \leq \alpha \\
                \d{y} = v, & \norm{v} \leq \beta 
            \end{cases}
        \end{gather*}
        Тут $\norm{z} = \sqrt{z_1^2 + z_2^2}$.
        Такі закони руху означають, що гравці рухаються з обмеженою швидкістю,
        але напрямок руху можуть змінювати довільно. Проінтегрувавши рівняння, можна явно записати траєкторії руху як
        \begin{gather*}
            x(t) = x(0) + \intl_0^t u(s) ds, \;
            y(t) = y(0) + \intl_0^t v(s) ds
        \end{gather*}
    \end{frame}
    \begin{frame}
        \frametitle{Приклад}
        \begin{block}{Умова}
            Нехай $u(t) = \begin{pmatrix} -\sin t \\ 2 \cos {2t}\end{pmatrix}$, 
            $v(t) = \begin{pmatrix} -\sqrt{2}\sin t \\ \sqrt{2}\cos t \end{pmatrix}$,
            $x(0) = \begin{pmatrix} 1 \\ 0 \end{pmatrix}$, 
            $y(0) = \begin{pmatrix} \sqrt{2} \\ 0 \end{pmatrix}$, 
            а гра триває до моменту $T = 2\pi$.
            Знайти значення функції виграшу мінімального результату з $H(x, y) = \norm{x - y}$.    
        \end{block}
        Знайдемо рівняння траєкторій:
        \begin{gather*}
            x(t) = \begin{pmatrix} 1 \\ 0 \end{pmatrix} +
            \intl_0^t \begin{pmatrix} -\sin s \\ 2 \cos {2s} \end{pmatrix} ds = 
            \begin{pmatrix} 1 \\ 0 \end{pmatrix} +
            \l.\begin{pmatrix} \cos s \\ \sin{2s} \end{pmatrix}\r|_0^t = 
            \begin{pmatrix} \cos t \\ \sin{2t} \end{pmatrix}
        \end{gather*}
        \begin{gather*}
            y(t) = \begin{pmatrix} \sqrt{2} \\ 0 \end{pmatrix} +
            \intl_0^t \begin{pmatrix} -\sqrt{2}\sin s \\ \sqrt{2}\cos s  \end{pmatrix} ds = 
            \begin{pmatrix} \sqrt{2}\cos t \\ \sqrt{2}\sin t \end{pmatrix}
        \end{gather*}
    \end{frame}
    \begin{frame}
        \frametitle{Приклад}

        На декартовій площині ці траекторії матимуть вигляд:
        \begin{center}
            \begin{tikzpicture}[scale=0.9]
                \begin{axis}
                    [axis lines = center,
                    axis equal,
                    trig format plots=rad,
                    xmin=-1.5, xmax=1.5, ymin=-1.5, ymax=1.5,
                    legend pos = outer north east]
                    \addplot [
                        domain=0:2*pi,
                        samples=100,
                        color=red,
                        decoration={markings, mark=between positions 0.05 and 0.1 step 2em with {\arrow [scale=1.5]{stealth}}
                        }, postaction=decorate, forget plot
                    ] ({cos(x)}, {sin(2*x)});
                    \addlegendimage{red}
                    \addlegendentry{$x(t)$}
                    \addplot [
                        domain=0:2*pi,
                        samples=100,
                        color=green,
                        decoration={markings, mark=between positions 0.05 and 0.1 step 2em with {\arrow [scale=1.5]{stealth}}
                        }, postaction=decorate, forget plot
                    ] ({sqrt(2)*cos(x)}, {sqrt(2)*sin(x)});
                    \addlegendimage{green}
                    \addlegendentry{$y(t)$}
                \end{axis}
            \end{tikzpicture}
        \end{center}
        Значення $K = \underset{0 \leq t \leq 2\pi}{\min} \norm{x(t) - y(t)}$ можна знайти чисельно:
        $K \approx 0.282394$ при $t \approx 0.850448$.
    \end{frame}
    \begin{frame}
        \frametitle{Простий рух в $\R^n$}
    
        Тепер розглянемо гру переслідування вже не на площині $\R^2$, а в $\R^n$:
        \begin{gather*}
            \begin{cases}
                \d{x} = u, & \norm{u} \leq \alpha \\
                \d{y} = v, & \norm{v} \leq \beta \\
                x(0) = x_0, \; y(0) = y_0
            \end{cases}
        \end{gather*}
        Тут усі величини є $n$-вимірними векторами, і, як раніше, $x(t)$ --- траєкторія
        руху переслідувача $P$, $y(t)$ --- утікача $E$.
    
        Нехай $\alpha > \beta$, тобто, переслідувач може рухатися швидше за утікача. 
        Тоді можна довести, що яку б стратегію $v(t)$ не обрав утікач $E$, переслідувач $P$ наздожене його не пізніше,
        ніж за $\frac{\norm{x_0 - y_0}}{\alpha - \beta}$, використовуючи стратегію $u(t) = - \frac{\alpha}{\norm{x(t) - y(t)}} (x(t) - y(t))$,
        причому переслідування буде найдовшим, якщо $E$ обере <<раціональну>> стратегію $v(t) = - \frac{\beta}{\norm{x(t) - y(t)}} (x(t) - y(t))$
        (див. повний текст).

    \end{frame}
    \begin{frame}
        \frametitle{Приклад}
    
        \begin{block}{Умова}
            Нехай $\alpha = 3, \beta = 1$, гра починається з $x_0 = \begin{pmatrix}
                0 \\ 0
            \end{pmatrix}$ та $y_0 = \begin{pmatrix}
                1 \\ 1
            \end{pmatrix}$, утікач обирає <<нераціональне>> керування $\d{y} = -\beta\begin{pmatrix}
                \cos t \\ \sin t
            \end{pmatrix}$, інший гравець обирає керування вказане вище.
        \end{block}
    \end{frame}
    \begin{frame}
        \frametitle{Приклад}
    
        Розв'яжемо чисельно (методом Рунге-Кутта) відповідну систему диференціальних рівнянь і подивимося на графіки $x(t)$ та $y(t)$ в залежності від часу.
        Отримаємо такі траекторії руху:
        \begin{center}
            \resizebox{220pt}{!}{
                % This file was created by tikzplotlib v0.9.8.
\begin{tikzpicture}

\begin{axis}[
legend cell align={left},
legend style={
  fill opacity=0.8,
  draw opacity=1,
  text opacity=1,
  at={(0.03,0.97)},
  anchor=north west,
  draw=white!80!black
},
tick align=outside,
tick pos=left,
x grid style={white!69.0196078431373!black},
xmin=-0.05, xmax=1.05,
xtick style={color=black},
y grid style={white!69.0196078431373!black},
ymin=-0.05, ymax=1.05,
ytick style={color=black}
]
\addplot [semithick, red, opacity=0.5]
table {%
0 0
0.0105932851190288 0.0106198960414178
0.021159700236 0.02126652698891
0.0316988822130722 0.0319401168028786
0.0422104569586439 0.0426408957383119
0.0526940389199339 0.0533691006111101
0.063149230544578 0.0641249750791759
0.0735756217088786 0.0749087699392946
0.0839727891101328 0.0857207434409194
0.0943402956202228 0.0965611616180674
0.104677689597386 0.10743029864064
0.114984504152788 0.118328437186596
0.125260256368191 0.129255868836524
0.135504446460648 0.140212894492321
0.145716556889743 0.151199824821814
0.155896051402429 0.162216980731347
0.166042374010034 0.173264693868552
0.176154947891387 0.184343307157712
0.18623317421542 0.195453175370385
0.196276430875824 0.206594665734204
0.206284071129573 0.217768158583058
0.216255422130141 0.228974048052206
0.226189783345249 0.240212742822208
0.236086424847735 0.251484666916021
0.245944585466824 0.262790260553995
0.25576347078549 0.27412998107211
0.265542250967865 0.285504303909306
0.275280058398616 0.296913723670478
0.284975985113851 0.308358755272424
0.294629080000491 0.319839935180906
0.304238345737879 0.331357822747954
0.313802735451851 0.342913001659649
0.323321149047308 0.35450608150589
0.33279242918045 0.3661376994851
0.34221535682614 0.377808522258513
0.351588646389225 0.38951924797057
0.360910940300707 0.401270608454232
0.370180803030418 0.413063371642535
0.379396714436786 0.42489834421074
0.388557062361119 0.436776374476913
0.397660134358122 0.448698355592838
0.406704108435419 0.460665229061943
0.415687042651996 0.472677988626568
0.424606863397761 0.484737684573505
0.433461352142544 0.496845428514657
0.442248130401335 0.509002398708968
0.450964642611308 0.521209846002988
0.459608136552474 0.533469100480782
0.468175640864278 0.545781578930001
0.476663939110216 0.558148793250353
0.48506953971549 0.570572359954255
0.493388640940294 0.583054010938084
0.501617089841821 0.595595605737447
0.509750333905566 0.608199145522706
0.517783363668394 0.620866789143687
0.525710644180233 0.633600871597323
0.533526032512506 0.646403925371912
0.541222677652617 0.659278705220098
0.548792897926292 0.672228217033434
0.556228029414238 0.685255751638129
0.563518236447625 0.698364924506905
0.57065227181929 0.711559722584032
0.577617169257636 0.724844559637692
0.584397843028219 0.73822434174943
0.590976557642789 0.751704544632362
0.597332211745117 0.765291304218482
0.603439349177112 0.778991520856151
0.6092667572795 0.792812974293959
0.614775418271778 0.806764438434008
0.619915403195263 0.820855764084858
0.624620946917299 0.835097844281568
0.62880218948818 0.849502230521717
0.632330290465487 0.864079734662894
0.63500785433867 0.878835891932277
0.636501291401439 0.89375519881629
0.63614554409209 0.908731860363251
0.631944396868688 0.922867110567458
0.625929499266278 0.924530136153144
0.622769953808564 0.923630192306444
0.62011865828777 0.921051836867392
0.61397871158503 0.918559919021901
0.609776714095582 0.916617683163057
0.605177239047701 0.914527148052339
0.600499725338173 0.912535706382767
0.595954498228315 0.910554551401068
0.591429392662837 0.908522481361473
0.586888339848174 0.906461894742526
0.582356182099426 0.904385286905881
0.577840487902751 0.902287207448624
0.573335443928958 0.900165031053234
0.568839706082846 0.898020142061739
0.564354775835445 0.895853159303068
0.559881042790697 0.893663816386726
0.555418294919473 0.891452049749428
0.550966593143991 0.889217991261226
0.54652612372493 0.886961721221984
0.542097006328689 0.884683277857244
0.537679334444396 0.882382713150513
0.533273217198133 0.880060088906855
0.528878768681292 0.877715464186126
0.524496098906028 0.875348896596565
0.520125316544535 0.872960445108438
0.515766530865101 0.870550169668123
0.51141985103896 0.868118130568892
0.507085385725658 0.865664388556938
0.502763243241337 0.863189004969141
0.498453531642635 0.860692041702603
0.494156358682207 0.858173561182354
0.489871831788038 0.855633626367367
0.48560005807084 0.853072300755728
0.481341144325069 0.850489648381112
0.477095197023838 0.847885733809705
0.472862322315515 0.845260622139085
0.468642626021641 0.842614378996908
0.464436213634427 0.83994707053915
0.460243190313987 0.837258763448382
0.456063660885681 0.834549524932132
0.45189772983752 0.831819422721224
0.447745501317565 0.829068525068072
0.443607079131315 0.826296900744975
0.439482566739108 0.823504619042395
0.435372067253538 0.82069174976723
0.43127568343688 0.817858363241065
0.427193517698515 0.815004530298416
0.423125672092373 0.812130322284957
0.419072248314384 0.80923581105574
0.41503334769993 0.806321068973394
0.411009071221317 0.803386168906319
0.406999519485247 0.800431184226865
0.403004792730304 0.797456188809494
0.39902499082445 0.794461257028937
0.395060213262523 0.791446463758333
0.391110559163759 0.788411884367356
0.387176127269302 0.785357594720332
0.383257015939745 0.782283671174345
0.379353323152669 0.779190190577322
0.375465146500188 0.776077230266117
0.371592583186516 0.772944868064577
0.367735730025535 0.769793182281592
0.363894683438374 0.766622251709144
0.360069539450996 0.763432155620331
0.356260393691802 0.760222973767389
0.352467341389237 0.756994786379699
0.348690477369413 0.753747674161775
0.344929896053731 0.750481718291256
0.34118569145653 0.747197000416866
0.33745795718273 0.743893602656383
0.333746786425492 0.740571607594578
0.330052271963894 0.737231098281154
0.326374506160603 0.73387215822867
0.322713580959573 0.730494871410454
0.319069587883744 0.727099322258498
0.315442618032753 0.723685595661356
0.311832762080657 0.720253776962015
0.308240110273667 0.716803951955762
0.304664752427892 0.713336206888044
};
\addlegendentry{$x(t)$}
\addplot [very thick, green, dashed]
table {%
1 1
0.995000020833306 0.999987500026042
0.990000166665831 0.999950000416665
0.985000562493669 0.999887502109359
0.980001333306663 0.999800006666578
0.975002604085282 0.999687516275703
0.970004499797498 0.999550033748987
0.965007145395656 0.999387562523488
0.960010665813357 0.999200106660978
0.95501518596233 0.998987670847842
0.950020830729311 0.998750260394966
0.94502772497292 0.998487881237598
0.940035993520542 0.998200539935204
0.935045761165203 0.997888243671301
0.930057152662452 0.997551000253279
0.925070292727241 0.997188818112207
0.92008530603081 0.996801706302619
0.915102317197565 0.99638967450229
0.910121450801969 0.995952733011993
0.905142831365422 0.995490892755244
0.90016658335315 0.995004165278025
0.895192831171095 0.994492562748496
0.890221699162801 0.993956097956696
0.885253311606311 0.993394784314214
0.880287792711055 0.992808635853865
0.875325266614745 0.992197667229327
0.870365857380277 0.991561893714786
0.865409688992622 0.990901331204546
0.860456885355733 0.990215996212635
0.855507570289442 0.989505905872392
0.850561867526368 0.98877107793604
0.845619900708823 0.988011530774237
0.840681793385719 0.987227283375624
0.835747669009483 0.986418355346345
0.830817650932967 0.985584766909558
0.825891862406366 0.98472653890493
0.820970426574137 0.983843692788118
0.816053466471919 0.982936250630228
0.811141105023458 0.982004235117266
0.806233465037536 0.981047669549573
0.801330669204896 0.980066577841237
0.796432840095178 0.9790609845195
0.791540100153855 0.978030914724143
0.786652571699171 0.976976394206857
0.781770376919083 0.9758974493306
0.776893637868206 0.974794107068938
0.772022476464762 0.973666395005369
0.767157014487533 0.972514341332637
0.762297373572814 0.971337974852023
0.757443675211375 0.970137324972629
0.752596040745423 0.968912421710638
0.747754591365567 0.967663295688568
0.742919448107789 0.966389978134506
0.738090731850419 0.965092500881323
0.733268563311111 0.963770896365883
0.728453063043828 0.96242519762823
0.723644351435826 0.961055438310762
0.718842548704645 0.959661652657392
0.714047774895102 0.958243875512688
0.709260149876294 0.956802142321004
0.704479793338596 0.955336489125596
0.699706824790673 0.953846952567717
0.69494136355649 0.952333569885703
0.69018352877233 0.950796378914042
0.685433439383814 0.94923541808243
0.68069121414293 0.947650726414804
0.675956971605061 0.946042343528375
0.671230830126025 0.944410309632631
0.666512907859113 0.942754665528334
0.661803322752135 0.9410754526065
0.657102192544474 0.939372712847366
0.65240963476414 0.937646488819336
0.647725766724833 0.935896823677921
0.643050705523011 0.934123761164659
0.638384568034959 0.932327345606019
0.633727470913873 0.930507621912299
0.629079530586937 0.928664635576494
0.624440863252417 0.926798432673169
0.619811584876756 0.924909059857296
0.615191811191671 0.9229965643631
0.610581657691265 0.921060994002868
0.605981239629134 0.919102397165757
0.60139067201549 0.917120822816587
0.596810069614286 0.915116320494613
0.592239546940341 0.913088940312289
0.587679218256486 0.911038732954014
0.583129197570698 0.908965749674865
0.57858959863326 0.906870042299316
0.574060534933908 0.904751663219942
0.569542119698997 0.902610665396111
0.565034465888675 0.900447102352655
0.560537686194051 0.898261028178539
0.556051893034384 0.896052497525502
0.551577198554268 0.893821565606697
0.547113714620833 0.891568288195305
0.542661552820945 0.889292721623144
0.538220824458416 0.886994922779259
0.533791640551226 0.884674949108503
0.529374111828739 0.882332858610096
0.524968348728946 0.879968709836178
0.520574461395693 0.877582561890346
0.516192559675934 0.875174474426174
0.511822753116986 0.872744507645723
0.507465150963784 0.870292722298037
0.503119862156155 0.867819179677621
0.498786995326093 0.865323941622912
0.494466658795043 0.862807070514731
0.490158960571193 0.860268629274725
0.485864008346775 0.857708681363793
0.48158190949537 0.855127290780499
0.477312771069227 0.852524522059473
0.473056699796584 0.849900440269799
0.468813802079001 0.847255111013383
0.4645841839887 0.844588600423319
0.460367951265913 0.841900975162234
0.456165209316239 0.839192302420619
0.451976063208007 0.836462649915151
0.447800617669652 0.833712085887001
0.443638977087095 0.830940679100126
0.439491245501134 0.828148498839552
0.435357526604842 0.82533561490964
0.431237923740976 0.822502097632341
0.427132539899394 0.81964801784544
0.423041477714478 0.816773446900783
0.418964839462568 0.813878456662493
0.414902727059411 0.810963119505177
0.410855242057602 0.80802750831211
0.406822485644058 0.80507169647342
0.402804558637478 0.802095757884249
0.398801561485828 0.799099766942907
0.394813594263829 0.796083798549011
0.390840756670453 0.793047928101614
0.386883148026433 0.789992231497319
0.382940867271779 0.786916785128382
0.379014012963305 0.783821665880802
0.375102683272164 0.780706951132399
0.371206975981395 0.777572718750879
0.367326988483476 0.774419047091889
0.363462817777894 0.771246014997057
0.359614560468714 0.768053701792018
0.355782312762169 0.764842187284437
0.351966170464252 0.76161155176201
0.348166228978321 0.758361875990455
0.344382583302717 0.755093241211499
0.340615328028383 0.751805729140841
0.336864557336506 0.74849942196611
0.333130364996157 0.745174402344815
0.32941284436195 0.741830753402272
0.325712088371708 0.738468558729531
0.322028189544138 0.735087902381284
0.318361239976518 0.731688868873762
0.314711331342396 0.728271543182628
0.311078554889299 0.724836010740845
0.307463001436448 0.721382357436546
0.303864761372492 0.717910669610882
0.300283924653244 0.714421034055869
};
\addlegendentry{$y(t)$}
\end{axis}

\end{tikzpicture}

            }
        \end{center}
    \end{frame}
    \begin{frame}
        \frametitle{Лінійна диференціальна гра}
    
        \begin{block}{Означення}
            \emph{Лінійною диференціальною грою} називається гра з фазовим простором
            $\R^n$, що описується рівнянням
            \begin{gather*}
                \begin{cases}
                    \d{z}(t) = A z(t) - u(t) + v(t) \\
                    z(0) = z_0 \\
                    u \in U, \; v \in V
                \end{cases}
            \end{gather*}
            де $A$ --- деяка стала матриця порядку $n\times n$, $U$ та $V$ --- опуклі компактні
            підмножини $\R^n$. Також задано матрицю $\pi$, що є матрицею проекції на ортогональне доповнення до термінальної множини.
        \end{block}
    \end{frame}
    \begin{frame}
        \frametitle{Лінійна диференціальна гра}
    
        Ця гра відбувається таким чином: в кожний момент часу $t$ утікач $E$ знає параметри гри
        $\l(A, U, V, z_0, \pi \r)$ та обирає своє керування $v(t) \in V$, повідомляючи про свій вибір
        переслідувача $P$, який, в свою чергу, обирає керування $u(t) \in U$.
        Якщо існує такий момент часу $T > 0$, коли переслідувач $P$ за будь-яких дій
        утікача $E$ забезпечує виконання умови $\pi z(\tau) = 0$ для деякого
        $\tau \in [0; T]$, то кажуть, що в переслідувач наздоганяє утікача.
        Отримаємо умови, за яких це відбувається. Для цього треба ввести декілька нових означень.
    
    \end{frame}
    \begin{frame}
        \frametitle{Операції над множинами}
    
        \begin{block}{Означення 1}
            \emph{Сумою множин (за Мінковським)} $A$ і $B$ називається множина
            $C = A + B = \l\{ a + b : a \in A, b \in B\r\}$.
        \end{block}

        \begin{block}{Означення 2}
            \emph{Різницею множин (за Мінковським)} $A$ і $B$ називається найбільша така множина
            $C = A \setdif B$, що $B + C \subset A$
        \end{block}

        \begin{block}{Означення 3}
            \emph{Добутком} множини $A$ на число $\lambda \in \R$ називається множина
            $\lambda \cdot A = \l\{\lambda \cdot a : a \in A \r\}$.
        \end{block}
    \end{frame}
    \begin{frame}
        \frametitle{Приклад}
    
        Нехай $B_{r}(a) = \l\{x \in \R^n : \norm{x - a} \leq r \r\}$ --- куля
        радіуса $r$ з центром в точці $a$. Для
        $r \in \R$ та $a \in \R^n$ має місце
        $r\cdot B_{1}(0) + \{a\} = B_{|r|}(a)$.
        Сумою двох куль $B_{r_1}(a_1)$ та $B_{r_2}(a_2)$ є множина
        \begin{gather*}
            M = B_{r_1}(a_1) + B_{r_2}(a_2) = \l\{ 
                x_1 + x_2 : \norm{x_1 - a_1} \leq r_1, \norm{x_2 - a_2} \leq r_2    
            \r\}
        \end{gather*}
        Для $x = x_1 + x_2 \in M$: $\norm{(x_1 + x_2) - (a_1 + a_2)} \leq \norm{x_1 - a_1} + \norm{x_2 - a_2} \leq r_1 + r_2$,
        тобто $M = B_{r_1 + r_2}(a_1 + a_2)$.
        \begin{center}
            \begin{tikzpicture}[scale=0.55]
                \begin{axis}
                    [axis lines = center,
                    axis equal,
                    trig format plots=rad,
                    xmin=-1.5, xmax=1.5, ymin=-1, ymax=1.5,
                    legend pos = outer north east]
                    \draw [very thick] (1,1) circle (0.4);
                    \draw [very thick] (-1,-0.6) circle (0.2);
                    \draw [very thick] (0,0.4) circle(0.6);
                    \node [below] at (1,0.6) {$B_1$};
                    \node [left] at (-1.2,-0.5) {$B_2$};
                    \node [above left] at (-0.52,0.7) {$B_1 + B_2$};
                    \draw [dashed] (1,0) -- (1,1) -- (0,1);
                    \draw [dashed] (-1,0) -- (-1,-0.6) -- (0,-0.6);
                    \fill (1,1) circle (0.03);
                    \fill (-1,-0.6) circle (0.03);
                    \fill (0,0.4) circle (0.03);
                \end{axis}
            \end{tikzpicture}
        \end{center}
    
    \end{frame}
    \begin{frame}
        \frametitle{Інтеграл багатозначного відображення}
    
        \begin{block}{Означення}
            Нехай $W(t)$ --- неперервна функція з дійсним аргументом, значеннями якої
        є компактні підмножини $\R^n$ (\emph{багатозначне відображення}).
        \emph{Інтегралом} за проміжком $[a;b]$ від неї називається множина
        $\intl_a^b W(t) dt$, яку можна розуміти в сенсі ріманової суми
        $\underset{n\to\infty}{\lim} \suml_{i=0}^n \Delta t_i\cdot W(t_i^*)$,
        де $\l\{\Delta t_i\r\}_{i=1}^n$ --- довжини відрізків, на які розбивається $[a;b]$, $t_i^* \in \Delta t_i$ ---
        деякі точки з цих відрізків, а сума розуміється в сенсі суми множин за Мінковським.
            \end{block}
    
    \end{frame}
    \begin{frame}
        \frametitle{Умови того, що $P$ дожене $E$}
    
        Нехай у лінійній диференціальній грі виконуються дві умови:
        \begin{enumerate}
            \item Для всіх $t>0$: $W(t) = \pi e^{At}U \setdif \pi e^{At}V \neq \varnothing$.
            \item Існує такий момент часу $T_0$, що $\pi e^{AT_0}z_0 \in \intl_0^T W(T_0 - s)ds$.
        \end{enumerate}

        Можна довести, що в разі виконання цих умов переслідувач наздожене утікача.
    
        Розв'язок задачі Коші та його образ під дією $\pi$ мають вигляд
        \begin{gather*}
            z(t) = e^{A t} z_0 + \intl_0^t e^{A(t-s)} (- u(s) + v(s)) ds
        \end{gather*}
        \begin{gather*}
            \pi z(t) =  \pi e^{A t} z_0 + \intl_0^t \pi e^{A(t-s)} (- u(s) + v(s)) ds
        \end{gather*}
    \end{frame}
    \begin{frame}
        \frametitle{Метод Понтрягіна}
    
        Наведемо алгоритм застосування методу Понтрягіна:
    
        \begin{enumerate}
            \item Знайти множину $W(t) = \pi e^{At}U \setdif \pi e^{At}V$.
            \item Знайти множину $\Omega(t) = \intl_0^t W(s)ds$.
            \item Знайти $T_0$, для якого $\pi e^{A T_0} z_0 \in \Omega(T_0)$.
            \item Знайти функцію $w(t) \in W(t)$ таку, що $\pi e^{A T_0} z_0 = \intl_0^{T_0} w(s) ds$.
            \item Знайти керування $u(t)$ як розв'язок $\pi e^{A (T_0-s)} u(s) - \pi e^{A (T_0-s)} v(s) = w(T_0 - s) $ при заданому керуванні $v(t) \in V$.
            \item Знайти розв'язок задачі Коші $\d{z} = Az - u(t) + v(t), \; z(0) = z_0$ на відрізку $[0; T_0]$.
        \end{enumerate}
    \end{frame}
    \begin{frame}
        \frametitle{Контрольний метод Понтрягіна}
    
        \begin{block}{Умова}
            Нехай рух гравців в $\R^n$, $n\geq 2$, описується системою
            \begin{gather*}
                \begin{cases}
                    \dd{x} + \alpha \d{x} = a, & \norm{a} \leq \rho \\
                    \dd{y} + \beta \d{y} = b, & \norm{b} \leq \sigma
                \end{cases}
            \end{gather*}
            де $\alpha, \beta, \rho, \sigma$ --- додатні числа. Переслідувач наздоганяє утікача, якщо $x=y$.
            Ця система описує рух точки одиничної маси під дією сили-керування з урахуванням тертя,
            що лінійно залежить від швидкості.
        \end{block}

        Перейдемо до системи диференціальних рівнянь першого порядку за допомогою замін $z^1 = x - y$, $z^2 = \d{x}$, $z^3 = \d{y}$:
        \begin{gather*}
            \begin{cases}
                \d{z}^1 = z^2 - z^3 \\
                \d{z}^2 = -\alpha z^2 + a \\
                \d{z}^3 = -\beta z^3 + b
            \end{cases}
        \end{gather*}
    \end{frame}
    \begin{frame}
        \frametitle{Контрольний приклад Понтрягіна}
    
        Керування $u$ та $v$ задаються формулами $u = (0, -a, 0)^T$, $v = (0, 0, b)^T$, тому
        $U = \l\{(0, -a, 0)^T : \norm{a} \leq \rho \r\}$, $V = \l\{(0, 0, b)^T : \norm{b} \leq \sigma \r\}$.
        Оператор $\pi$ задано як $\pi: (z^1, z^2, z^3)^T \mapsto (z^1, 0, 0)^T$, а матриця $A$
        дорівнює $\begin{pmatrix}
            0 & 1 & -1 \\
            0 & -\alpha & 0 \\
            0 & 0 & -\beta
        \end{pmatrix}$. 
        
        Знайдемо $e^{A t}$ (за допомогою перетворення Лапласа):
    
        \begin{gather*}
            e^{At} = \Lap{(pI - A)^{-1}} \Leftrightarrow (pI - A)^{-1} = \LapInv{e^{At}} \\
            pI - A = \begin{pmatrix}
                p & -1 & 1 \\
                0 & p+\alpha & 0 \\
                0 & 0 & p + \beta \\
            \end{pmatrix}
        \end{gather*}
    \end{frame}
    \begin{frame}
        \frametitle{Контрольний приклад Понтрягіна}

        \begin{gather*}
            (pI - A)^{-1} = \frac{1}{p(p+\alpha)(p+\beta)}\begin{pmatrix}
                (p+\alpha)(p+\beta) & (p+\beta) & -(p+\alpha) \\
                0 & p(p+\beta) & 0 \\
                0 & 0 & p(p+\alpha)
            \end{pmatrix} = \\ =
            \begin{pmatrix}
                \frac{1}{p} & \frac{1}{p(p+\alpha)} & - \frac{1}{p(p+\beta)} \\
                0 & \frac{1}{p+\alpha} & 0 \\
                0 & 0 & \frac{1}{p+\beta}
            \end{pmatrix} \Rightarrow
            e^{At} = \begin{pmatrix}
                1 & \frac{1 - e^{-\alpha t}}{\alpha} & -\frac{1 - e^{-\beta t}}{\beta} \\
                0 & e^{-\alpha t} & 0 \\
                0 & 0 & e^{-\beta t}
            \end{pmatrix}
        \end{gather*}   

        Тепер можна записати $\pi e^{At}(z^1, z^2, z^3) = z^1 + \frac{1 - e^{-\alpha t}}{\alpha} z^2 - \frac{1 - e^{-\beta t}}{\beta}z^3$, звідки
        \begin{gather*}
            \pi e^{At} U = \l\{\frac{1 - e^{-\alpha t}}{\alpha} \cdot (-a) : \norm{a} \leq \rho\r\} \\ 
            \pi e^{At} V = \l\{-\frac{1 - e^{-\beta t}}{\beta} \cdot b : \norm{b} \leq \sigma\r\}
        \end{gather*}
    \end{frame}
    \begin{frame}
        \frametitle{Контрольний приклад Понтрягіна}
    
        Отже, $\pi e^{At} U$ --- куля з радіусом $\frac{1 - e^{-\alpha t}}{\alpha}\rho$ і центром в нулі,
        а $\pi e^{At} V$ --- куля з радіусом $\frac{1 - e^{-\beta t}}{\beta}\sigma$ і центром в нулі, тому
        $W(t) = \pi e^{At}U \setdif \pi e^{At}V$ --- куля з радіусом 
        $\frac{1 - e^{-\alpha t}}{\alpha}\rho - \frac{1 - e^{-\beta t}}{\beta}\sigma$ і центром теж в нулі.
        Радіус $W(t)$ буде додатнім при $\rho > \sigma$ та $\frac{\rho}{\alpha} > \frac{\sigma}{\beta}$.
        
        $\Omega(t) = \intl_0^t W(s) ds$ буде кулею
        з радіусом:
        \begin{gather*}
            \intl_0^t \left(\frac{1 - e^{-\alpha s}}{\alpha}\rho - \frac{1 - e^{-\beta s}}{\beta} \sigma\right) ds = 
            \rho \intl_0^t \frac{1 - e^{-\alpha s}}{\alpha} ds - \sigma \intl_0^t \frac{1 - e^{-\beta s}}{\beta} ds =
        \end{gather*}
        \begin{gather*}
            = \frac{\rho}{\alpha^2}\l(\alpha t + e^{-\alpha t} - 1\r) - 
            \frac{\sigma}{\beta^2}\l(\beta t + e^{-\beta t} - 1\r) = r(t)
        \end{gather*}
    
    \end{frame}
    \begin{frame}
        \frametitle{Контрольний приклад Понтрягіна}
    
        Далі необхідно знайти таке (найменше) значення $T_0$, для якого точка $\pi e^{A T_0}(z^1_0, z^2_0, z^3_0)$ належить
        $\Omega(T_0)$. З геометричних міркувань це буде найменший корінь рівняння
        \begin{gather*}
            \frac{\rho}{\alpha^2}\l(\alpha t + e^{-\alpha t} - 1\r) - 
            \frac{\sigma}{\beta^2}\l(\beta t + e^{-\beta t} - 1\r) = \norm{\pi e^{A t} z_0}
        \end{gather*}

    \end{frame}
    \begin{frame}
        \frametitle{Контрольний приклад Понтрягіна}
    
        Наступний крок будемо проводити на конкретному прикладі. Нехай гра відбувається на площині з 
        $\alpha=1$, $\beta=2$, $\rho=2$, $\sigma=1$ та
        початковими умовами $x(0) = \begin{pmatrix}
            3 \\ 2
        \end{pmatrix}, \d{x}(0) = \begin{pmatrix}
            1 \\ 1
        \end{pmatrix}, y(0) = \begin{pmatrix}
            1 \\ 0
        \end{pmatrix}, \d{y}(0) = \begin{pmatrix}
            0 \\ 1
        \end{pmatrix}$. Чисельно можна знайти значення $T_0 \approx 3.715$.
        \begin{center}
            \resizebox{160pt}{!}{
                \input{../code/pontr_method_plot.tex}
            }
        \end{center}
    \end{frame}
    \begin{frame}
        \frametitle{Метод розв’язуючих функцій}
    
        Метод розв’язуючих функцій належить А.О. Чикрію.
        Розглядається не просто лінійна диференціальна гра, а більш загальна
        \emph{квазілінійна} гра виду:
    
        \begin{gather*}
            \begin{cases}
                \d{z} = A z + \vf(u, v) \\
                z(0) = z_0 \\
                z \in \R^n, u \in U, v \in V
            \end{cases}
        \end{gather*}
        де $\vf(u, v) : U\times V \to \R^n$ --- неперервна за обома змінними функція.
        
        \begin{block}{Зауваження}
            У випадку $\vf(u,v) = -u + v$ отримуємо лінійну диференціальну гру.
        \end{block}
    \end{frame}
    \begin{frame}
        \frametitle{Метод розв’язуючих функцій}
    
        Термінальна множина має вид $M^* = M^0 + M$, де
        $M^0$ --- деякий лінійний підпростір $\R^n$, а $M$ --- компактна підмножина ортогонального доповнення $M^0$,
        $\pi$ --- проектор на $(M^0)^\perp$.

        Нехай $\vf(U, v) = \l\{\vf(u,v) : u \in U\r\}$ для фіксованої $v \in V$,
        $W(t, v) = \pi e^{At} \vf(U, v)$,
        $W(t) = \bigcap\limits_{v \in V} W(t, v), t\geq 0$. 
        \begin{block}{Зауваження}
            У випадку $\vf(u,v) = -u + v$ $W(t) = \pi e^{At}U \setdif \pi e^{At} V$.
        \end{block}
    
    \end{frame}
    \begin{frame}
        \frametitle{Позначення}
    
        Вводяться позначення:
        \begin{gather*}
            \gamma(t) \in W(t) \text{ --- вимірна функція} \\
            \xi(t, z, \gamma(\cdot)) = \pi e^{A t} + \intl_0^t \gamma(\tau) d\tau \\
            \alpha(t, \tau, z, v, \gamma(\cdot)) = \\ = \sup\l\{ 
                \alpha \geq 0 : \l[ W(t-\tau, v) - \gamma(t-\tau)\r] \cap \alpha
                \l[M - \xi(t, z, \gamma(\cdot))\r] \neq \varnothing
            \r\} \\
            T(z, \gamma(\cdot)) = \inf \l\{ 
                t\geq 0: \intl_0^t \underset{v\in V}{\inf} \alpha(t, \tau, z, v, \gamma(\cdot)) d\tau \geq 1
            \r\}
        \end{gather*}

        \begin{block}{Означення}
            $\alpha(t, \tau, z, v, \gamma(\cdot))$ називається \emph{розв'язуючою функцією}.
        \end{block}
    \end{frame}
    \begin{frame}
        \frametitle{Метод розв’язуючих функцій}
    
        Можна довести наступне: якщо $W(t) \neq \varnothing$ для всіх $t\geq 0$,
        $M$ --- опукла множина, $T(z_0, \gamma_0(\cdot)) < +\infty$ для деякого початкового положення
        $z_0$ та деякої $\gamma_0(\cdot)$, то за час $T(z_0, \gamma_0(\cdot))$ гравці потрапляють у термінальну множину.

        Також можна довести, що якщо гра є лінійною,
        $W(t) = \pi e^{At}U \setdif \pi e^{At} V \neq \varnothing$, існує неперервна 
        $r(t): [0; +\infty) \to [0; +\infty)$ та число $l \geq 0$ такі, що
        $\pi e^{A t}U = r(t) S$, $M = l S$, де $S$ --- одинична куля в $(M^0)^\perp$ з центром в нулі, то
        при $\xi(t, z, \gamma(\cdot)) \notin l S$, розв'язуюча функція $\alpha$ може бути знайдена як найбільший додатний корінь квадратного рівняння
    
    \end{frame}
    \begin{frame}
        \frametitle{Контрольний приклад Понтрягіна}
    
        Як і у минулому розв’язку цього прикладу перейдемо від системи
    
        \begin{gather*}
            \begin{cases}
                \dd{x} + \alpha \d{x} = \rho u, & \norm{a} \leq 1 \\
                \dd{y} + \beta \d{y} = \sigma v, & \norm{b} \leq 1
            \end{cases}
        \end{gather*}

        до системи з $z_1 = x - y$, $z_2 = \d{x}$, $z_3 = \d{y}$:
        \begin{gather*}
            \begin{cases}
                \d{z}_1 = z_2 - z_3 \\
                \d{z}_2 = -\alpha z_2 + \rho u \\
                \d{z}_3 = -\beta z_3 + \sigma v
            \end{cases}
        \end{gather*}

        Термінальна множина $M^* = \l\{z: z_1 = 0 \r\} = M^0 + \{0\}$, ортогональне доповнення
        $(M^0)^\perp =\l\{z: z_2 = z_3 = 0 \r\}$, тому $\pi = \begin{pmatrix}
            I & 0 & 0 \\
            0 & 0 & 0 \\
            0 & 0 & 0
        \end{pmatrix}$, де $I$ та $0$ --- тотожній та нульовий оператор відповідно.

    \end{frame}
    \begin{frame}
        \frametitle{Контрольний приклад Понтрягіна}
    
        Оскільки у вихідній системі рівнянь $x, y \in \R^n$, то $z \in \R^{3n}$,
        то $\pi : \R^{3n} \to (M^0)^\perp$. Матриця системи
        $A = \begin{pmatrix}
            0 & E & -E \\
            0 & -\alpha E & 0 \\
            0 & 0 & -\beta E
        \end{pmatrix}$,
        $
        U = \l\{ \begin{pmatrix}0 \\ \rho u \\ 0 \end{pmatrix} : \norm{u} \leq 1\r\}
        $,
        $
        V = \l\{ \begin{pmatrix}0 \\ 0 \\ \sigma v \end{pmatrix} : \norm{v} \leq 1\r\}
        $.
        Аналогічно минулому прикладу,
        \begin{gather*}
            \pi e^{At} U = \frac{1 - e^{-\alpha t}}{\alpha} \rho S, \;
            \pi e^{At} V = \frac{1 - e^{-\beta t}}{\beta} \sigma S \\
            W(t) = \l(\frac{1 - e^{-\alpha t}}{\alpha} \rho - \frac{1 - e^{-\beta t}}{\beta} \sigma\r) S = \omega(t) S
        \end{gather*}
        Радіус цієї кулі невід'ємний при $\rho \geq \sigma$ та $\frac{\rho}{\alpha} \geq \frac{\sigma}{\beta}$.
    
    \end{frame}
    \begin{frame}
        \frametitle{Контрольний приклад Понтрягіна}
    
        Поклавши $\gamma(t) = 0$, отримаємо
        $\xi(t, z, 0) = z_1 + \frac{1 - e^{-\alpha t}}{\alpha} z_2 - \frac{1 - e^{-\beta t}}{\beta}z_3$.
        Ця задача задовольняє всі умови для пошуку розв'язуючої функції через:
        \begin{gather*}
            \norm{\frac{1 - e^{-\beta t}}{\beta} \sigma v - \alpha \cdot \xi(t, z, 0)} = \frac{1 - e^{-\alpha t}}{\alpha} \rho
        \end{gather*}
        Можна показати, що 
        \begin{gather*}
            \underset{\norm{v}\leq 1}{\min}{\alpha(t, \tau, z, v, 0)} = \frac{\omega(t-\tau)}{\norm{\xi(t, z, 0)}}
        \end{gather*}
        і мінімум досягається при $v = -\frac{\xi(t, z, 0)}{\norm{\xi(t, z, 0)}}$
    
    \end{frame}
    \begin{frame}
        \frametitle{Контрольний приклад Понтрягіна}
    
        Час, коли переслідувач наздожене утікача, визначається як
        \begin{gather*}
            T(z, 0) = \min \l\{ t\geq 0 : \intl_0^t \frac{\omega(t-\tau)}{\norm{\xi(t, z, 0)}} d\tau = 1 \r\}
        \end{gather*}
        або ж як найменший додатний корінь рівняння
        \begin{gather*}
            \norm{\xi(t, z, 0)} = \intl_0^t \l(\frac{1 - e^{-\alpha \tau}}{\alpha} \rho - \frac{1 - e^{-\beta \tau}}{\beta} \sigma\r) d\tau
        \end{gather*}
    
    \end{frame}
    \begin{frame}
        \frametitle{Контрольний приклад Понтрягіна}
    
        Розглянемо тепер конкретний приклад. Нехай ця гра відбувається на площині з 
        $\alpha=1$, $\beta=2$, $\rho=2$, $\sigma=1$ та
        початковими умовами $x(0) = \begin{pmatrix}
            3 \\ 2
        \end{pmatrix}, \d{x}(0) = \begin{pmatrix}
            1 \\ 1
        \end{pmatrix}, y(0) = \begin{pmatrix}
            1 \\ 0
        \end{pmatrix}, \d{y}(0) = \begin{pmatrix}
            0 \\ 1
        \end{pmatrix}$.
        Чисельно можна знайти значення $T_0 \approx 3.715$.
        \begin{center}
            \resizebox{160pt}{!}{
                % This file was created by tikzplotlib v0.9.8.
\begin{tikzpicture}

\definecolor{color0}{rgb}{0.12156862745098,0.466666666666667,0.705882352941177}
\definecolor{color1}{rgb}{1,0.498039215686275,0.0549019607843137}

\begin{axis}[
legend cell align={left},
legend style={
  fill opacity=0.8,
  draw opacity=1,
  text opacity=1,
  at={(0.97,0.03)},
  anchor=south east,
  draw=white!80!black
},
tick align=outside,
tick pos=left,
x grid style={white!69.0196078431373!black},
xmin=-0.204323093744256, xmax=4.29078496862938,
xtick style={color=black},
y grid style={white!69.0196078431373!black},
ymin=-0.220660969615349, ymax=4.63388036192233,
ytick style={color=black}
]
\addplot [draw=none, draw=gray, fill=gray, forget plot, mark=*]
table{%
x  y
3.7149653408046555 3.871012274588237
};
\addplot [semithick, color0]
table {%
0 0
0.0412773926756073 0.000851675551738981
0.0825547853512146 0.0034039613308894
0.123832178026822 0.00764899720016568
0.165109570702429 0.0135744146071706
0.206386963378036 0.021163901319297
0.247664356053644 0.0303977130192027
0.288941748729251 0.041253136310856
0.330219141404858 0.0537049073118953
0.371496534080466 0.0677255896639959
0.412773926756073 0.0832859154766874
0.45405131943168 0.100355092429385
0.495328712107288 0.118901079989231
0.536606104782895 0.138890837456842
0.577883497458502 0.160290546326418
0.619160890134109 0.183065809239396
0.660438282809717 0.20718182762035
0.701715675485324 0.232603559908876
0.742993068160931 0.259295862140481
0.784270460836539 0.287223612481875
0.825547853512146 0.316351821190482
0.866825246187753 0.346645727343524
0.908102638863361 0.378070883567689
0.949380031538968 0.410593229895515
0.990657424214575 0.444179157778245
1.03193481689018 0.478795565196554
1.07321220956579 0.514409903729416
1.1144896022414 0.550990218366914
1.155766994917 0.588505180784594
1.19704438759261 0.626924116734254
1.23832178026822 0.666217028148695
1.27959917294383 0.706354610505276
1.32087656561943 0.747308265944848
1.36215395829504 0.789050112598411
1.40343135097065 0.831552990533316
1.44470874364626 0.874790464693667
1.48598613632186 0.918736825175607
1.52726352899747 0.963367085147014
1.56854092167308 1.00865697669263
1.60981831434868 1.05458294483956
1.65109570702429 1.10112213999425
1.6923730996999 1.14825240900014
1.73365049237551 1.19595228500534
1.77492788505111 1.24420097631142
1.81620527772672 1.29297835435761
1.85748267040233 1.34226494097973
1.89876006307794 1.3920418950691
1.94003745575354 1.44229099874407
1.98131484842915 1.49299464313515
2.02259224110476 1.54413581387452
2.06386963378036 1.59569807637061
2.10514702645597 1.64766556094006
2.14642441913158 1.70002294786138
2.18770181180719 1.75275545240719
2.22897920448279 1.80584880990557
2.2702565971584 1.85928926087503
2.31153398983401 1.91306353627204
2.35281138250962 1.9671588428854
2.39408877518522 2.02156284890709
2.43536616786083 2.07626366970525
2.47664356053644 2.13124985382142
2.51792095321205 2.1865103692107
2.55919834588765 2.24203458974067
2.60047573856326 2.2978122819622
2.64175313123887 2.35383359216281
2.68303052391447 2.41008903371129
2.72430791659008 2.46656947469998
2.76558530926569 2.52326612588994
2.8068627019413 2.58017052896202
2.8481400946169 2.63727454507615
2.88941748729251 2.69457034373945
2.93069487996812 2.7520503919828
2.97197227264373 2.80970744384485
3.01324966531933 2.86753453016111
3.05452705799494 2.92552494865546
3.09580445067055 2.98367225433055
3.13708184334615 3.04197025015309
3.17835923602176 3.10041297802949
3.21963662869737 3.15899471006699
3.26091402137298 3.21770994011503
3.30219141404858 3.27655337558126
3.34346880672419 3.33551992951656
3.3847461993998 3.39460471296288
3.42602359207541 3.45380302755809
3.46730098475101 3.51311035839133
3.50857837742662 3.57252236710275
3.54985577010223 3.63203488522127
3.59113316277783 3.69164390773383
3.63241055545344 3.75134558688004
3.67368794812905 3.81113622616557
3.71496534080466 3.87101227458824
3.75624273348026 3.93097032107035
3.79752012615587 3.99100708909118
3.83879751883148 4.05111943151347
3.88007491150709 4.11130432559795
3.92135230418269 4.17155886819989
3.9626296968583 4.23188027114197
4.00390708953391 4.29226585675761
4.04518448220951 4.35271305359942
4.08646187488512 4.41321939230698
};
\addlegendentry{$\int_0^t \omega(\tau)d\tau$}
\addplot [semithick, color1]
table {%
0 2.82842712474619
0.0412773926756073 2.85773589965896
0.0825547853512146 2.88717896014367
0.123832178026822 2.91661288129743
0.165109570702429 2.94591661713905
0.206386963378036 2.9749883778964
0.247664356053644 3.00374293894893
0.288941748729251 3.03210932626525
0.330219141404858 3.06002882791448
0.371496534080466 3.08745328651118
0.412773926756073 3.11434363279201
0.45405131943168 3.14066862562369
0.495328712107288 3.16640376844449
0.536606104782895 3.19153037637054
0.577883497458502 3.21603477193383
0.619160890134109 3.2399075906765
0.660438282809717 3.26314318063888
0.701715675485324 3.28573908219011
0.742993068160931 3.30769557670555
0.784270460836539 3.32901529434074
0.825547853512146 3.34970287262879
0.866825246187753 3.36976465887628
0.908102638863361 3.38920845038574
0.949380031538968 3.40804326742015
0.990657424214575 3.42627915457361
1.03193481689018 3.44392700684284
1.07321220956579 3.46099841722672
1.1144896022414 3.47750554313052
1.155766994917 3.49346098923186
1.19704438759261 3.50887770478715
1.23832178026822 3.52376889363096
1.27959917294383 3.53814793535269
1.32087656561943 3.55202831633319
1.36215395829504 3.56542356949292
1.40343135097065 3.57834722174821
1.44470874364626 3.59081274829621
1.48598613632186 3.6028335329564
1.52726352899747 3.61442283388869
1.56854092167308 3.62559375408822
1.60981831434868 3.6363592161262
1.65109570702429 3.64673194066661
1.6923730996999 3.6567244283414
1.73365049237551 3.66634894461273
1.77492788505111 3.67561750729147
1.81620527772672 3.68454187641684
1.85748267040233 3.69313354623337
1.89876006307794 3.7014037390292
1.94003745575354 3.70936340062458
1.98131484842915 3.71702319732106
2.02259224110476 3.72439351414148
2.06386963378036 3.73148445420832
2.10514702645597 3.73830583912348
2.14642441913158 3.74486721022637
2.18770181180719 3.75117783061985
2.22897920448279 3.75724668786463
2.2702565971584 3.7630824972529
2.31153398983401 3.76869370558097
2.35281138250962 3.77408849534897
2.39408877518522 3.77927478932283
2.43536616786083 3.78426025540077
2.47664356053644 3.78905231173211
2.51792095321205 3.79365813204203
2.55919834588765 3.79808465112057
2.60047573856326 3.80233857043875
2.64175313123887 3.80642636385863
2.68303052391447 3.81035428340783
2.72430791659008 3.81412836509229
2.76558530926569 3.81775443472399
2.8068627019413 3.82123811374324
2.8481400946169 3.82458482501725
2.88941748729251 3.82779979859931
2.93069487996812 3.83088807743455
2.97197227264373 3.83385452300023
3.01324966531933 3.83670382087017
3.05452705799494 3.83944048619426
3.09580445067055 3.84206886908544
3.13708184334615 3.84459315990761
3.17835923602176 3.84701739445928
3.21963662869737 3.84934545904819
3.26091402137298 3.85158109545372
3.30219141404858 3.85372790577398
3.34346880672419 3.85578935715574
3.3847461993998 3.85776878640549
3.42602359207541 3.85966940448096
3.46730098475101 3.86149430086243
3.50857837742662 3.86324644780397
3.54985577010223 3.86492870446503
3.59113316277783 3.86654382092287
3.63241055545344 3.86809444206721
3.67368794812905 3.86958311137803
3.71496534080466 3.87101227458824
3.75624273348026 3.87238428323279
3.79752012615587 3.87370139808624
3.83879751883148 3.87496579249052
3.88007491150709 3.87617955557531
3.92135230418269 3.877344695373
3.9626296968583 3.8784631418307
4.00390708953391 3.87953674972151
4.04518448220951 3.88056730145759
4.08646187488512 3.88155650980735
};
\addlegendentry{$\left\Vert\xi(T_0, z, 0)\right\Vert$}
\end{axis}

\end{tikzpicture}

            }
        \end{center}
    
    \end{frame}
\end{document}