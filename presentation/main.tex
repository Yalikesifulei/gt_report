\documentclass[10pt,pdf]{beamer}

\usefonttheme[onlymath]{serif}
\usepackage{amsmath,amsfonts,amssymb,amsthm,mathtools}
\usepackage[T2A]{fontenc}
\usepackage[utf8]{inputenc}
\usepackage[english,ukrainian]{babel}
\usepackage{ragged2e}
\justifying
\addtobeamertemplate{navigation symbols}{}{%
    \usebeamerfont{footline}%
    \usebeamercolor[fg]{footline}%
    \hspace{1em}%
    \insertframenumber/\inserttotalframenumber
}

\usepackage{tikz}
\usepackage{pgfplots}
\usepgfplotslibrary{fillbetween}
\pgfplotsset{compat=1.17}

\newcommand{\R}{\mathbb{R}}
\newcommand{\vf}{\varphi}
\renewcommand{\d}[1]{\dot{#1}}
\newcommand{\dd}[1]{\ddot{#1}}
\newcommand{\E}{\mathcal{E}}
\newcommand{\T}{\mathcal{T}}
\newcommand{\bvec}[1]{\boldsymbol{\rm #1}}
\renewcommand{\l}{\left}
\renewcommand{\r}{\right}
\newcommand{\norm}[1]{\left\Vert #1 \right\Vert}
\newcommand{\intl}{\int\limits}
\newcommand{\suml}{\sum\limits}
\newcommand{\dotprod}[2]{\l< #1, #2\r>}
\newcommand{\Lap}[1]{\mathcal{L}\l\{#1\r\}}
\newcommand{\LapInv}[1]{\mathcal{L}^{-1}\l\{#1\r\}}
\def\setdif{\mathbin{\ooalign{\hss\raise1ex\hbox{.}\hss\cr
  \mathsurround=0pt$-$}}}

\usepackage{graphicx}
\graphicspath{ {./pics/} }

\usetheme{Luebeck}
\title{Основи диференціальних ігор}
\author{Н. Фордуй, О. Галганов}


\date{}
\begin{document}
    \begin{frame}
        \titlepage
    \end{frame}
    \begin{frame}
        \frametitle{Основні поняття}
    
        Рiшення, що їх приймають гравцi, полягають у виборi так званих \textbf{керувань}, вiд яких залежать \textbf{фазовi координати}: 
        їх значення у будь-який момент часу повнiстю визначає хiд гри, характеризуючи положення гравцiв у деякому просторi — \textbf{фазовому просторi}. 

        Поточнi значення фазових координат завжди вiдомi
        гравцям — тобто, це iгри з повною iнформацiєю. Невiдомим зазвичай є
        характер їх змiни: тобто, \textbf{керування} фазовими змiнними гравцями.
        \begin{block}{Приклад}
            Положення матеріальної точки на площині описується двома координатами $x_1$ та $x_2$. 
            Нехай швидкість руху точки є сталою $v$, а гравець обирає напрямок швидкості $\vf$ та може змінювати його у будь-який момент часу --- тобто, $\vf$
            є керуванням. Тоді рух точки описується системою диференціальних рівнянь
            \begin{gather*}
                \begin{cases}
                    \d{x_1} = v \cos \vf \\
                    \d{x_2} = v \sin \vf
                \end{cases}
            \end{gather*}
        \end{block}
    \end{frame}
    \begin{frame}
        \frametitle{Приклад}
    
        Геометричне положення автомобіля на декартовій площині описується трьома фазовими координатами:
        $x_1, x_2$ --- положення деякої точки автомобіля, $x_3$ --- кут, який утворює вісь вздовж автомобіля
        з деяким фіксованим напрямком --- наприклад, $x_1$.
        \begin{center}
            \begin{tikzpicture}
                \draw [->, thick] (-0.5, 0) -- (5, 0);
                \draw [->, thick] (0, -0.5) -- (0, 3);
                \node [below] at (5, 0) {$x_1$};
                \node [left] at (0, 3) {$x_2$};
                \filldraw [fill=lightgray, rounded corners, rotate around={-40:(1,1)}] (1, 1) rectangle (1.7, 2.4);
                \fill [rotate around={-40:(1,1)}] (1.35, 2.4) circle [radius=2pt];
                \draw [->, dashed, rotate around={-40:(1,1)}] (1.35, 0) -- (1.35, 3);
                \draw (0.83, 0) arc (0:44:0.2);
                \draw (0.93, 0) arc (0:45:0.3);
                \node [right] at (0.9, 0.22) {$x_3$};
            \end{tikzpicture}
        \end{center}
    \end{frame}
    \begin{frame}
        \frametitle{Продовження приклада}

        Нехай $A$ --- максимальне можливе прискорення автомобіля, тоді прискорення може набувати значень
        $A \vf_1$, де $\vf_1 \in [0; 1]$ і знаходиться під контролем гравця-водія. Можна ввести ще одну фазову координату $x_4$ --- швидкість автомобіля.
        Таким чином, можна ввести кривину як ще одну фазову координату $x_5$
        (фізично --- це кут повороту передніх коліс), керуванням якої є $W \vf_2$, де $\vf_2 \in [-1; 1]$, а $W$ --- максимальна швидкість зміни $x_5$.

        Система задає рух автомобіля у деякій диференціальній грі:
        \begin{gather*}
            \begin{cases}
                \d{x_1} = x_4 \cos{x_3} \\
                \d{x_2} = x_4 \sin{x_3} \\
                \d{x_3} = x_4 x_5 \\
                \d{x_4} = A \vf_1, \; \vf_1 \in [0; 1] \\
                \d{x_5} = W \vf_2, \; \vf_2 \in [-1; 1]
            \end{cases}
        \end{gather*}
    \end{frame}
    \begin{frame}
        \frametitle{Опис руху}
    
        Вважаємо, що гра відбувається у \emph{фазовому просторі} $\E$ --- деякій області в $\R^n$ та на її межі.
        Рух точки $x = \l(x_1, x_2, ..., x_n \r)$ у фазовому просторі описується системою диференціальних рівнянь
        \begin{gather*}\label{eq_1}
            \begin{cases}
                \d{x_1}(t) = f_1(x_1(t), ..., x_n(t), u_1(x, t), ..., u_P(x, t), v_1(x, t), ..., w_E(x, t)) \\
                \d{x_2}(t) = f_2(x_1(t), ..., x_n(t), u_1(x, t), ..., u_P(x, t), v_1(x, t), ..., w_E(x, t)) \\
                \dots \\
                \d{x_n}(t) = f_n(x_1(t), ..., x_n(t), u_1(x, t), ..., u_P(x, t), v_1(x, t), ..., w_E(x, t)) \\
                x_1(0) = x_1^0, x_2(0) = x_2^0, ..., x_n(0) = x_n^0
            \end{cases}
        \end{gather*}
        або, коротше,
        \begin{gather*}\label{eq_2}
            \begin{cases}
                \d{x}(t) = {f}(x(t), u(x, t), v(x, t)) \\
                x(0) = x_0
            \end{cases}
        \end{gather*}
        Ці рівняння називаються \emph{рівняннями руху}. Функції $f_j$ є заданими та вважаються достатньо гладкими.
    \end{frame}
    \begin{frame}
        \frametitle{Приклад}
    
        Якщо позначити через $(x_P, y_P)$ координати гравця $P$, через $(x_E, y_E)$ --- гравця $E$, через $w_P$ та $w_E$ їх сталі швидкості руху, 
        а керування напрямком швидкості через $u(t)$ та $v(t)$ відповідно, то отримаємо такі рівняння руху:
        \begin{gather*}
            \begin{cases}
                \d{x_P}(t) = w_P \cos u(t) \\
                \d{y_P}(t) = w_P \sin u(t) \\
                \d{x_E}(t) = w_E \cos v(t) \\
                \d{y_E}(t) = w_E \sin v(t) \\
                (x_P(0), y_P(0)) = (x_P^0, y_P^0) \\
                (x_E(0), y_E(0)) = (x_E^0, y_E^0)
            \end{cases}
        \end{gather*}
        Такий рух називається <<переслідуванням на площині з простим рухом гравців>>.
    \end{frame}
    \begin{frame}
        \frametitle{Виграші}
    
        Мета диференціальної гри визначається виграшем, який залежить від траєкторій гравців. 
        Позначимо ці траєкторії як функції від часу як $x(t)$ та $y(t)$. 
        Зауважимо, що диференціальні ігри є \emph{антагоністичними} (або ж, \emph{іграми з нульовою сумою}).

        Якщо гра триває деякий заздалегідь визначений час $T$, то виграш гравця $E$ визначається
        як $H(x(t), y(T))$, де $H : \R^n \times \R^n \to \R$ --- деяка функція (нагадаємо, що розмірність $\E$ --- $n$).

        \begin{block}{Приклади виграшів}
            \begin{enumerate}
                \item $H(x(T), y(T)) = \norm{x(T) - y(T)}$
                \item $H(x(T), y(T)) = \underset{0 \leq t \leq T}{\min} \norm{x(t) - y(t)}$
                \item $H(x(T), y(T)) = t_* = \min \l\{ t \geq 0 : (x(t), y(t)) \in \T\r\}$
            \end{enumerate}
        \end{block}
    \end{frame}
    \begin{frame}
        \frametitle{Приклад}
    
        Розглянемо переслідування на площині з простим рухом, що описується системою
        \begin{gather*}
            \begin{cases}
                \d{x_1} = u_1, \d{x_2} = u_2, \; u_1^2 + u_2^2 \leq \alpha^2 \\
                \d{y_1} = v_1, \d{y_2} = v_2, \; v_1^2 + v_2^2 \leq \beta^2 \\
                x_1(0) = x_1^0, x_2(0) = x_2^0, y_1(0) = y_1^0, y_2(0) = y_2^0
            \end{cases}
        \end{gather*}

        Якщо $\alpha > \beta$, то гравець $P$ може гарантувати
        \begin{gather*}
            \forall l \geq 0 : \min \l\{ t \geq 0 : \norm{P(t) - E(t)} \leq l\r\} < +\infty
        \end{gather*}

        Якщо $\alpha \leq \beta$, то в разі $\norm{P(0) - E(0)} > l$ для всіх $l \geq 0$ гравець $E$, рухаючись від $P$ по прямій з максимальною швидкістю,
        зможе уникнути захоплення гравцем $P$.
    \end{frame}
    \begin{frame}
        \frametitle{Поняття стратегії}
    
        \begin{block}{Означення}
            \emph{Стратегіями} у диференціальній грі є вибір керувань $u$ та $v$ як функцій від часу $t$ та
            фазових координат $x$ у системі рівнянь руху
            $$
            \begin{cases}
                \d{x}(t) = {f}(x(t), u(x, t), v(x, t)) \\
                x(0) = x_0
            \end{cases}
            $$
        \end{block}
        Керування вважаються кусково-гладкими як компроміс між забезпеченням існування розв'язку, 
        (його може не існувати у класі неперервних функцій) та його єдиності (вона може порушуватися, якщо не вимагати неперервності розв'язку).

        Позначатимемо через $\rm P$ та $\rm E$ множини кусково-неперервних стратегій (керувань) гравців $P$ та $E$. 
    \end{frame}
    \begin{frame}
        \frametitle{Ситуація}
    
        Надалі для спрощення розглядатимемо не один вектор $x$, а два вектори $x$ та $y$, що відповідатимуть руху кожного з гравців. Тоді
        систему можна записати як
        \begin{gather}\label{eq_3}
            \begin{cases}
                \d{x}(t) = f(x(t), u(x, y, t)) \\
                \d{y}(t) = g(x(t), v(x, y, t)) \\
                x(0) = x_0, y(0) = y_0
            \end{cases}
        \end{gather}

        \begin{block}{Означення}
            Набір $S = \l\{x_0, y_0, u(\cdot), v(\cdot) \r\}$, де $x_0, y_0$ --- початкові умови, а $u \in \rm P$, $v \in \rm E$ --- керування, 
            називається \emph{ситуацією} в диференціальній грі.
        \end{block}

        
    \end{frame}
    \begin{frame}
        \frametitle{Умова існування та єдиності траекторій}
    
        Якщо розглядати траєкторії, що залежать лише від часу $t$ та накладати на $f$ та $g$ умови
        обмеженості та ліпшицевості по $x$ та $y$, тобто
        \begin{gather*}
            \norm{f(x_1, u) - f(x_2, u)} \leq \alpha \cdot \norm{x_1 - x_2}, \\
            \norm{g(y_1, v) - g(y_2, v)} \leq \beta \cdot \norm{y_1 - y_2},
        \end{gather*}
        то за теоремою про існування та єдиність роз'язку задачі Коші, для кожної ситуації $S$ буде існувати єдина пара траєкторій $x(t), y(t)$,
        для якої 
        \begin{gather*}
            \begin{cases}
                \d{x}(t) = f(x(t), u(t)) \\
                \d{y}(t) = g(y(t), v(t)) \\
                x(0) = x_0, y(0) = y_0
            \end{cases}
        \end{gather*}
    \end{frame}
    \begin{frame}
        \frametitle{Формальне визначення виграшів та їх види}

        \begin{block}{Означення}
            Користуючись означенням ситуації, можна ввести \emph{виграш} в ситуації $S = \l\{x_0, y_0, u(\cdot), v(\cdot) \r\}$
            як функцію $K(x_0, y_0, u(\cdot), v(\cdot))$.
        \end{block}

        Наведемо строгі означення 4 видів виграшів.

        \begin{block}{Означення}
            \begin{enumerate}
                \item \emph{Термінальний виграш.} Задано деяке число $t>0$ та неперервна по $x$ та $y$ функція $H(x, y)$. Виграш в ситуації $\l\{x_0, y_0, u(\cdot), v(\cdot) \r\}$
                визначається як:
                \begin{gather*}
                    K(x_0, y_0, u(\cdot), v(\cdot)) = H(x(T), y(T))
                \end{gather*}
            \end{enumerate}
        \end{block}
    \end{frame}
    \begin{frame}
        \frametitle{Види виграшів}
        
        \begin{block}{Означення}
            \begin{enumerate}
                \setcounter{enumi}{1}
                \item \emph{Мінімальний результат}. Задано деяке число $t>0$ та неперервна по $x$ та $y$ функція $H(x, y)$. Виграш в ситуації $\l\{x_0, y_0, u(\cdot), v(\cdot) \r\}$
                визначається як
                \begin{gather*}
                    K(x_0, y_0, u(\cdot), v(\cdot)) = \underset{0 \leq t \leq T}{\min} H(x(t), y(t))
                \end{gather*}

                \item  \emph{Інтегральний виграш}. Нехай $\T$ --- деяка підмножина $\R^n \times \R^n$, $H(x, y)$ --- неперервна функція. Нехай в ситуації $\l\{x_0, y_0, u(\cdot), v(\cdot) \r\}$
                $t_*$ --- перший момент потрапляння траєкторії $(x(t), y(t))$ на $\T$.
                Тоді
                \begin{gather*}
                    K(x_0, y_0, u(\cdot), v(\cdot)) = \intl_0^{t_*} H(x(t), y(t)) dt
                \end{gather*}
                де при $t_* = +\infty$ покладається $K = +\infty$.

            \end{enumerate}
        \end{block}
    
    \end{frame}
    \begin{frame}
        \frametitle{Види виграшів}
    
        \begin{block}{Означення}
            \begin{enumerate}
                \setcounter{enumi}{3}
                \item \emph{Якісний виграш}. Нехай $\T$ та $\mathcal{L}$ --- деякі підмножини $\R^n \times \R^n$, а $t_*$ --- перший момент потрапляння траєкторії $(x(t), y(t))$ на $\T$
                в ситуації $\l\{x_0, y_0, u(\cdot), v(\cdot) \r\}$. Тоді
                \begin{gather*}
                    K(x_0, y_0, u(\cdot), v(\cdot)) = \begin{cases}
                        1, & \text{ якщо } (x(t_*), y(t_*)) \in \mathcal{L} \\
                        0, & \text{ якщо } t_* = +\infty \\
                        -1, & \text{ якщо } (x(t_*), y(t_*)) \notin \mathcal{L} \\
                    \end{cases}
                \end{gather*}
            \end{enumerate}
        \end{block}
    \end{frame}
    \begin{frame}
        \frametitle{Нормальна форма диференціальної гри}
    
        Нарешті, можна дати означення нормальної форми диференціальної гри.

        \begin{block}{Означення}
            Нормальною формою диференціальної гри $\Gamma (x_0, y_0)$, заданої на просторі стратегій $\mathrm{P} \times \mathrm{E}$, називається система
            \begin{gather}
                \Gamma (x_0, y_0) = \l<x_0, y_0, \mathrm {P}, \mathrm{E}, K(x_0, y_0, u(\cdot), v(\cdot)) \r>
            \end{gather}
            де $K(x_0, y_0, u(\cdot), v(\cdot))$ --- функція виграшу, визначена будь-який з чотирьох способів вище.
        \end{block}

        Кожній парі $(x_0, y_0) \in \R^n \times \R^n$ відповідає своя гра в нормальній формі, тобто, фактично,
        визначається двопараметрична сім'я ігор, що залежать від $(x_0, y_0)$.
    \end{frame}
\end{document}