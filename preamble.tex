\usepackage[14pt]{extsizes}
\usepackage[T2A]{fontenc}
\usepackage[utf8]{inputenc}
\usepackage[english,ukrainian]{babel}

\usepackage[a4paper, top=20mm, bottom=20mm, left=30mm, right=20mm]{geometry}
\usepackage{amsmath,amsfonts,amssymb,amsthm,mathtools}
\usepackage{graphicx}
\usepackage{enumitem}
\usepackage{verbatim}
\usepackage{anyfontsize}
\usepackage{indentfirst}
\usepackage{titlesec}

\usepackage[psdextra]{hyperref}
\hypersetup{unicode=true}
\hypersetup{
    colorlinks=true,
    linkcolor=blue,
}

\setlist[enumerate]{nosep}
\graphicspath{{pics/}}
\DeclareGraphicsExtensions{.png}

\usepackage{tikz}
\usepackage{pgfplots}
\usepgfplotslibrary{fillbetween}
\pgfplotsset{compat=1.17}

\usetikzlibrary{patterns}
\usetikzlibrary{shapes.geometric}
\usetikzlibrary{arrows.meta}
\usetikzlibrary{shapes.geometric}
\usetikzlibrary{positioning,arrows}
\usetikzlibrary{calc}
\tikzset{
	block/.style={
		draw, 
		rectangle, 
		minimum height=1cm, 
		minimum width=3cm, align=center
	}, 
	line/.style={->,>=stealth'}}
\tikzset{>=latex}

\newcommand\centerarc{} % just for safety
\def\centerarc[#1]#2(#3)#4(#5:#6:#7)% [draw options] (center) (initial angle:final angle:radius)
  {\draw[#1]($(#3)+({#7*cos(#5)},{#7*sin(#5)})$)arc(#5:#6:#7);}

\newcommand{\R}{\mathbb{R}}
\newcommand{\vf}{\varphi}
\renewcommand{\d}[1]{\dot{#1}}
\newcommand{\E}{\mathcal{E}}
\newcommand{\T}{\mathcal{T}}
\newcommand{\bvec}[1]{\boldsymbol{\rm #1}}
\renewcommand{\l}{\left}
\renewcommand{\r}{\right}
\newcommand{\norm}[1]{\Vert #1 \Vert}
\newcommand{\intl}{\int\limits}


\counterwithout{equation}{chapter} 
\counterwithin*{equation}{section}

\newtheoremstyle{exampstyle}
{3pt} % Space above
{3pt} % Space below
{} % Body font
{} % Indent amount
{\bfseries} % Theorem head font
{.} % Punctuation after theorem head
{.5em} % Space after theorem head
{} % Theorem head spec (can be left empty, meaning `normal')

\theoremstyle{exampstyle}
\newtheorem{definition}{Означення}
\newtheorem*{definition*}{Означення}
\newtheorem{example}{Приклад}[chapter]
\newtheorem*{example*}{Приклад}
\newtheorem{exercise}{Задача}
\newtheorem*{exercise*}{Задача}

\theoremstyle{remark}
\newtheorem*{remark}{Зауваження}